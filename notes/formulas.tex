\documentclass[a4paper]{article}
\usepackage[english]{babel}
\usepackage[utf8]{inputenc}
\usepackage{textcomp}
\usepackage{amsmath}
\usepackage{gensymb}
\usepackage{physics}
\usepackage{graphicx}
\usepackage[colorinlistoftodos]{todonotes}
\usepackage{xcolor}
\usepackage{array}
\usepackage{tabularx}
\usepackage{tikz}
\usepackage{pgfplots}
\usepackage{framed}
\usepackage{xfrac}
\usepackage[most]{tcolorbox}
\usepackage{fix-cm}
\usepackage{cancel}
\usepackage[margin=0.1in]{geometry}
\usepackage{multicol}
\usepackage{setspace}
\usetikzlibrary{quotes,angles}
\usetikzlibrary{decorations.pathreplacing}
\usetikzlibrary{calc}
\usepgfplotslibrary{fillbetween}

\let\phi\varphi
\let\bf\textbf
\colorlet{shadecolor}{orange!15}
\pgfplotsset{compat=1.18}
\newcommand\der[2]{\frac{d #1}{d #2}}
\newcommand\Deltat{\Delta t}
\newcommand\rads{\text{ rad\;s}^{-1}}
\newcommand\radss{\text{ rad\;s}^{-2}}
\newcommand\rad{\text{ rad}}
\newcommand\s{\text{ s}}
\newcommand\m{\text{ m}}
\newcommand\km{\text{ km}}
\newcommand\J{\text{ J}}
\newcommand\Nm{\text{ Nm}}
\newcommand\ms{\text{ ms}^{-1}}
\newcommand\mss{\text{ ms}^{-2}}
\newcommand\kg{\text{ kg}}
\newcommand\kgms{\text{ kg\;ms}^{-1}}
\newcommand\kgmm{\text{ kg\;m}^{2}}
\newcommand\kgmms{\text{ kg\;m}^2\text{s}^{-1}}
\newcommand{\AxisRotator}[1][rotate=0]{%
    \tikz [x=0.25cm,y=0.60cm,line width=.2ex,-stealth,#1] \draw (0,0) arc (-150:150:1 and 1);%
}
\def\centerarc[#1](#2)(#3:#4:#5){\draw[#1] ($(#2)+({#5*cos(#3)},{#5*sin(#3)})$) arc (#3:#4:#5)}
% Syntax: \centerarc[draw options] (center) (initial angle:final angle:radius);
\doublespacing

\title{PHY 2048 Formulas}

\begin{document}

\begin{center}{\huge{\bf{PHY 2048 Formulas}}}
\end{center}

\begin{multicols*}{3}

    \tikzstyle{mybox} = [draw=black,fill=white,very thick,rectangle,rounded corners, inner sep=10pt, inner ysep=10pt]
    \tikzstyle{fancytitle} = [fill=black,text=white,font=\bfseries]

    %%% DOT AND CROSS PRODUCTS %%%
    \begin{tikzpicture}
        \node[mybox] (box){
        \begin{minipage}{0.25\textwidth}
            $\vec{A} \cdot \vec{B} = AB\cos(\phi)$\\
            $\vec{A} \times \vec{B} = AB\sin(\phi)$
        \end{minipage}
        };
        \node[fancytitle,right=10pt] at (box.north west) {Dot and Cross Product};
    \end{tikzpicture}

    %%% TRANSLATIONAL KINEMATICS %%%
    \begin{tikzpicture}
        \node[mybox] (box){
        \begin{minipage}{0.25\textwidth}
            $\displaystyle x = x_0 + vt$\\
            $\displaystyle v = v_{0} + at$\\
            $\displaystyle x = x_0 + v_{0}t + \frac{1}{2}at^2$\\
            $\displaystyle v^2 = v_0^2 + 2a(\Delta x)$
        \end{minipage}
        };
        \node[fancytitle,right=10pt] at (box.north west) {Translational Kinematics};
    \end{tikzpicture}

    %%% ROTATIONAL KINEMATICS %%%
    \begin{tikzpicture}
        \node[mybox] (box){
        \begin{minipage}{0.25\textwidth}
            $\displaystyle \theta_f = \theta_0 + \omega t$\\
            $\displaystyle \omega_f = \omega_0 + \alpha t$\\
            $\displaystyle \theta_f = \theta_0 + \omega_0t + \frac{1}{2}\alpha t^2$\\
            $\displaystyle \omega_f^2 = \omega_0^2 + 2\alpha(\Delta\theta)$
        \end{minipage}
        };
        \node[fancytitle,right=10pt] at (box.north west) {Rotational Kinematics};
    \end{tikzpicture}

    %%% PROJECTILE MOTION %%%
    \begin{tikzpicture}
        \node[mybox] (box){
        \begin{minipage}{0.25\textwidth}
            \bf{Time of flight:} $T = \frac{2v_0\sin(\theta_0)}{g}$\\
            \bf{Trajectory:}\\ $\displaystyle y = x\tan(\theta) - \bigg[\frac{g}{2(v_0\cos(\theta))^2}\bigg]x^2$\vspace{1.5mm}\\
            \bf{Range:} $\displaystyle \frac{v_0^2\sin(2\theta_0)}{g}$
        \end{minipage}
        };
        \node[fancytitle,right=10pt] at (box.north west) {Projectile Motion};
    \end{tikzpicture}

    %%% UNIFORM CIRCULAR MOTION %%%
    \begin{tikzpicture}
        \node[mybox] (box){
        \begin{minipage}{0.25\textwidth}
            $\displaystyle a_c = \frac{v_T^2}{r}$\vspace{2mm}\\
            $\displaystyle \omega = \frac{2\pi}{T}$\vspace{2mm}\\
            $\displaystyle T = \frac{2\pi}{V}$\vspace{2mm}\\
            $\displaystyle r = \frac{v^2}{g}$
        \end{minipage}
        };
        \node[fancytitle,right=10pt] at (box.north west) {Uniform Circular Motion};
    \end{tikzpicture}

    %%% ORBITAL MOTION %%%
    \begin{tikzpicture}
        \node[mybox] (box){
        \begin{minipage}{0.25\textwidth}
            $\displaystyle F_{c} = F_g$\vspace{2mm}\\
            $\displaystyle v = \sqrt{\frac{GM}{r}}$\vspace{2mm}\\
            $\displaystyle v = \frac{2\pi r}{T}$
        \end{minipage}
        };
        \node[fancytitle,right=10pt] at (box.north west) {Orbital Motion};
    \end{tikzpicture}

    %%% NEWTON'S LAWS %%%
    \begin{tikzpicture}
        \node[mybox] (box){
        \begin{minipage}{0.25\textwidth}
            1. $v =$ constant when $F_{net} = 0$\\
            2. $\displaystyle F_{net} = ma = \der{p}{t} = \der{}{t}(mv)$\\
            3. $\vec{F}_{AB} = -\vec{F}_{BA}$
        \end{minipage}
        };
        \node[fancytitle,right=10pt] at (box.north west) {Newton's Laws};
    \end{tikzpicture}

    %%% COMMON FORCES %%%
    \begin{tikzpicture}
        \node[mybox] (box){
        \begin{minipage}{0.25\textwidth}
            $N = mg\cos(\theta)$\\
            $F_{sp} = -k\Delta x$\\
            $f_s \leq \mu_sN$\\
            $f_k = \mu_kN$\\
            $\displaystyle F_c = ma_c = m\frac{v^2}{r} = mr\omega^2$
        \end{minipage}
        };
        \node[fancytitle,right=10pt] at (box.north west) {Common Forces};
    \end{tikzpicture}

    %%% TERMINAL VELOCITY %%%
    \begin{tikzpicture}
        \node[mybox] (box){
        \begin{minipage}{0.25\textwidth}
            $\displaystyle v_T = \sqrt{\frac{2mg}{\rho CA}}$
        \end{minipage}
        };
        \node[fancytitle,right=10pt] at (box.north west) {Terminal Velocity};
    \end{tikzpicture}

    %%% WORK %%%
    \begin{tikzpicture}
        \node[mybox] (box){
        \begin{minipage}{0.25\textwidth}
            $\displaystyle W_{AB} = \int\limits_{A}^{B}\vec{F} \cdot d\vec{r}$\\
            $W = \vec{F} \cdot \vec{d} = Fd\cos(\theta)$\\
            $W_g = -mg\Delta y$\\
            $\displaystyle W_{sp} = \frac{1}{2}k(x_f^2 - x_0^2)$\\
            $W_f = \mu_kNd = \mu_kmgd$
        \end{minipage}
        };
        \node[fancytitle,right=10pt] at (box.north west) {Work};
    \end{tikzpicture}

    %%% WORK ENERGY THEOREM %%%
    \begin{tikzpicture}
        \node[mybox] (box){
        \begin{minipage}{0.25\textwidth}
            \vspace{1mm} $W_{net} = \Delta K = K_f - K_0$
        \end{minipage}
        };
        \node[fancytitle,right=10pt] at (box.north west) {Work-Energy Theorem};
    \end{tikzpicture}
    %%% ENERGY %%%

    \begin{tikzpicture}
        \node[mybox] (box){
        \begin{minipage}{0.25\textwidth}
            $\displaystyle K = \frac{1}{2}mv^2 = \frac{p^2}{2m}$\\
            $U_g = mgh$\\
            $\displaystyle U_{sp} = \frac{1}{2}k(x_f^2 - x_0^2)$
        \end{minipage}
        };
        \node[fancytitle,right=10pt] at (box.north west) {Energy};
    \end{tikzpicture}
\end{multicols*}
\end{document}

%\begin{tikzpicture}
%    \node[mybox] (box){
%    \begin{minipage}{0.25\textwidth}
%        
%    \end{minipage}
%    };
%    \node[fancytitle,right=10pt] at (box.north west) {};
%\end{tikzpicture}