\documentclass[a4paper]{article}
\usepackage[english]{babel}
\usepackage[utf8]{inputenc}
\usepackage{textcomp}
\usepackage{amsmath}
\usepackage{gensymb}
\usepackage{physics}
\usepackage{graphicx}
\usepackage{xcolor}
\usepackage{array}
\usepackage{tikz}
\usepackage{pgfplots}
\usepackage{xfrac}
\usepackage[most]{tcolorbox}
\usepackage{fix-cm}
\usepackage{cancel}
\usepackage[margin=0.1in]{geometry}
\usepackage{multicol}
\usepackage{setspace}
\usepackage{pagecolor}
\usepackage{esint}

\let\phi\varphi
\let\bf\textbf
\colorlet{shadecolor}{orange!15}
\pgfplotsset{compat=1.18}
\newcommand\der[2]{\frac{d #1}{d #2}}
\newcommand\Deltat{\Delta t}
\newcommand\rads{\text{ rad\;s}^{-1}}
\newcommand\radss{\text{ rad\;s}^{-2}}
\newcommand\rad{\text{ rad}}
\newcommand\s{\text{ s}}
\newcommand\m{\text{ m}}
\newcommand\km{\text{ km}}
\newcommand\J{\text{ J}}
\newcommand\Nm{\text{ Nm}}
\newcommand\ms{\text{ ms}^{-1}}
\newcommand\mss{\text{ ms}^{-2}}
\newcommand\kg{\text{ kg}}
\newcommand\kgms{\text{ kg\;ms}^{-1}}
\newcommand\kgmm{\text{ kg\;m}^{2}}
\newcommand\kgmms{\text{ kg\;m}^2\text{s}^{-1}}
\newcommand{\AxisRotator}[1][rotate=0]{%
    \tikz [x=0.25cm,y=0.60cm,line width=.2ex,-stealth,#1] \draw (0,0) arc (-150:150:1 and 1);%
}
\def\centerarc[#1](#2)(#3:#4:#5){\draw[#1] ($(#2)+({#5*cos(#3)},{#5*sin(#3)})$) arc (#3:#4:#5)}
% Syntax: \centerarc[draw options] (center) (initial angle:final angle:radius);
\doublespacing

\title{PHY 2048 Formulas}

\begin{document}
\definecolor{bg0}{HTML}{282828}
\definecolor{fg0}{HTML}{fbf1c7}
\definecolor{fg1}{HTML}{ebdbb2}
\definecolor{gbaqua}{HTML}{8ec07c}
\definecolor{gbred}{HTML}{fb4934}
\definecolor{gbgreen}{HTML}{b8bb26}
\definecolor{gbyellow}{HTML}{fabd2f}
\definecolor{gbblue}{HTML}{83a598}
\definecolor{gbpurple}{HTML}{d3869b}
\definecolor{gborange}{HTML}{fe8019}
\color{bg0}
\pagecolor{fg0}

\begin{center}{\huge{\bf{PHY 2048 Formulas}}}
\end{center}

\begin{center}
\begin{multicols*}{3}

    \tikzstyle{mybox} = [draw=bg0,fill=fg0,very thick,rectangle,rounded corners, inner sep=10pt, inner ysep=10pt]
    \tikzstyle{fancytitle} = [draw=bg0,fill=gbaqua,text=bg0,font=\bfseries,rounded corners]

    %%% DOT AND CROSS PRODUCTS %%%
    \begin{tikzpicture}
        \node[mybox] (box){
        \begin{minipage}{0.25\textwidth}
            $\vec{A} \cdot \vec{B} = AB\cos(\phi)$\\
            $\vec{A} \times \vec{B} = AB\sin(\phi)$
        \end{minipage}
        };
        \node[fancytitle,right=10pt,fill=gbred] at (box.north west) {Dot and Cross Product};
    \end{tikzpicture}

    %%% TRANSLATIONAL KINEMATICS %%%
    \begin{tikzpicture}
        \node[mybox] (box){
        \begin{minipage}{0.25\textwidth}
            $\displaystyle x = x_0 + vt$\\
            $\displaystyle v = v_{0} + at$\\
            $\displaystyle x = x_0 + v_{0}t + \frac{1}{2}at^2$\\
            $\displaystyle v^2 = v_0^2 + 2a(\Delta x)$
        \end{minipage}
        };
        \node[fancytitle,right=10pt,fill=gbgreen] at (box.north west) {Translational Kinematics};
    \end{tikzpicture}

    %%% PROJECTILE MOTION %%%
    \begin{tikzpicture}
        \node[mybox] (box){
        \begin{minipage}{0.25\textwidth}
            \bf{Time of flight:} $T = \frac{2v_0\sin(\theta_0)}{g}$\\
            \bf{Trajectory:}\\ $\displaystyle y = x\tan(\theta) - \bigg[\frac{g}{2(v_0\cos(\theta))^2}\bigg]x^2$\vspace{2mm}\\
            \bf{Range:} $\displaystyle \frac{v_0^2\sin(2\theta_0)}{g}$
        \end{minipage}
        };
        \node[fancytitle,right=10pt,fill=gbyellow] at (box.north west) {Projectile Motion};
    \end{tikzpicture}

    %%% UNIFORM CIRCULAR MOTION %%%
    \begin{tikzpicture}
        \node[mybox] (box){
        \begin{minipage}{0.25\textwidth}
            \vspace{2mm}$\displaystyle a_c = \frac{v_T^2}{r}$ \hspace{15mm} $\displaystyle T = \frac{2\pi}{V}$\vspace{2mm}\\
            $\displaystyle \omega = \frac{2\pi}{T}$ \hspace{15mm} $\displaystyle r = \frac{v^2}{g}$
        \end{minipage}
        };
        \node[fancytitle,right=10pt,fill=gbblue] at (box.north west) {Uniform Circular Motion};
    \end{tikzpicture}

    %%% ORBITAL MOTION %%%
    \begin{tikzpicture}
        \node[mybox] (box){
        \begin{minipage}{0.25\textwidth}
            \vspace{2mm}$\displaystyle v = \sqrt{\frac{GM}{r}}$\vspace{2mm}\\
            $\displaystyle v = \frac{2\pi r}{T}$
        \end{minipage}
        };
        \node[fancytitle,right=10pt,fill=gbpurple] at (box.north west) {Orbital Motion};
    \end{tikzpicture}

    %%% TERMINAL VELOCITY %%%
    \begin{tikzpicture}
        \node[mybox] (box){
        \begin{minipage}{0.25\textwidth}
            \vspace{1.5mm}$\displaystyle v_T = \sqrt{\frac{2mg}{\rho CA}}$
        \end{minipage}
        };
        \node[fancytitle,right=10pt] at (box.north west) {Terminal Velocity};
    \end{tikzpicture}

    %%% NEWTON'S LAWS %%%
    \begin{tikzpicture}
        \node[mybox] (box){
        \begin{minipage}{0.25\textwidth}
            1. $v =$ constant when $F_{net} = 0$\\
            2. $\displaystyle F_{net} = ma = \der{p}{t} = \der{}{t}(mv)$\\
            3. $\vec{F}_{AB} = -\vec{F}_{BA}$
        \end{minipage}
        };
        \node[fancytitle,right=10pt,fill=gborange] at (box.north west) {Newton's Laws};
    \end{tikzpicture}

    %%% COMMON FORCES %%%
    \begin{tikzpicture}
        \node[mybox] (box){
        \begin{minipage}{0.25\textwidth}
            $N = mg\cos(\theta)$\\
            $F_{sp} = -k\Delta x$\\
            $f_s \leq \mu_sN$\\
            $f_k = \mu_kN$\\
            $\displaystyle F_c = ma_c = m\frac{v^2}{r} = mr\omega^2$
        \end{minipage}
        };
        \node[fancytitle,right=10pt,fill=gbred] at (box.north west) {Common Forces};
    \end{tikzpicture}

    %%% CONSERVATIVE FORCE %%%
    \begin{tikzpicture}
        \node[mybox] (box){
        \begin{minipage}{0.25\textwidth}
            \vspace{2mm}$\displaystyle \der{F_x}{y} = \der{F_y}{x}$
        \end{minipage}
        };
        \node[fancytitle,right=10pt,fill=gbgreen] at (box.north west) {Conservative Force};
    \end{tikzpicture}

    %%% WORK %%%
    \begin{tikzpicture}
        \node[mybox] (box){
        \begin{minipage}{0.25\textwidth}
            $\displaystyle W_{AB} = \int_{A}^{B}\vec{F} \cdot d\vec{r}$\vspace{2mm}\\
            $W = \vec{F} \cdot \vec{d} = Fd\cos(\theta)$\\
            $W_g = -mg\Delta y$\\
            $\displaystyle W_{sp} = \frac{1}{2}k(x_f^2 - x_0^2)$\\
            $W_f = \mu_kNd = \mu_kmgd$
        \end{minipage}
        };
        \node[fancytitle,right=10pt,fill=gbyellow] at (box.north west) {Work};
    \end{tikzpicture}

    %%% ENERGY %%%
    \begin{tikzpicture}
        \node[mybox] (box){
        \begin{minipage}{0.25\textwidth}
            $\displaystyle K_t = \frac{1}{2}mv^2 = \frac{p^2}{2m}$\\
            $\displaystyle K_r = \frac{1}{2}I\omega^2$\\
            $U_g = mgh$\\
            $\displaystyle U_{sp} = \frac{1}{2}k(x_f^2 - x_0^2)$\\
            $\Delta K_{AB} = -\Delta U_{AB}$\\
            $\displaystyle W = \oint \vec{F} \cdot d\vec{r} = 0\ $ (closed path)
        \end{minipage}
        };
        \node[fancytitle,right=10pt,fill=gbblue] at (box.north west) {Energy};
    \end{tikzpicture}

    %%% WORK ENERGY THEOREM %%%
    \begin{tikzpicture}
        \node[mybox] (box){
        \begin{minipage}{0.25\textwidth}
            \vspace{1mm} $W_{net} = \Delta K = K_f - K_0$\\
            $-W_{net} = \Delta U = U_f - U_0$
        \end{minipage}
        };
        \node[fancytitle,right=10pt,fill=gbpurple] at (box.north west) {Work-Energy Theorem};
    \end{tikzpicture}

    %%% CONSERVATION OF ENERGY %%%
    \begin{tikzpicture}
        \node[mybox] (box){
        \begin{minipage}{0.25\textwidth}
            \vspace{1mm}\bf{With conservative forces:}\\
            $K_A + U_A =  K_B + U_B$\\
            $W_{nc} = \Delta(K + U) = \Delta E$
        \end{minipage}
        };
        \node[fancytitle,right=10pt] at (box.north west) {Conservation of Energy};
    \end{tikzpicture}

    %%% MOMENTUM & IMPULSE %%%
    \begin{tikzpicture}
        \node[mybox] (box){
        \begin{minipage}{0.25\textwidth}
            \vspace{1mm}$\vec{p} = m\vec{v}$\\
            $\vec{J} = \Delta p = m\Delta v$\\
            $\displaystyle \vec{J} = \vec{F}_{avg}\Delta t \to \vec{F}_{avg} = \frac{\Delta\vec{p}}{\Delta t}$\\
            $\displaystyle \vec{J} = \int_{t_0}^{t_f}\vec{F}(t)dt$\vspace{2mm}\\
            $\displaystyle \vec{F} = \der{\vec{p}}{t}$
        \end{minipage}
        };
        \node[fancytitle,right=10pt,fill=gborange] at (box.north west) {Momentum \& Impulse};
    \end{tikzpicture}

    %%% ROLLING MOTION %%%
    \begin{tikzpicture}
        \node[mybox] (box){
        \begin{minipage}{0.25\textwidth}
            %\bf{(Without slipping)}\\
            $\displaystyle a_{cm} = \frac{mg\sin(\theta)}{1 + (\sfrac{I_{cm}}{r^2})}$
        \end{minipage}
        };
        \node[fancytitle,right=10pt,fill=gbred] at (box.north west) {Rolling Motion, No Slipping};
    \end{tikzpicture}

    %%% CONSERVATION OF MOMENTUM %%%
    \begin{tikzpicture}
        \node[mybox] (box){
        \begin{minipage}{0.25\textwidth}
            If:\\
            1. $\displaystyle \bigg[\frac{dm}{dt}\bigg]_{sys} = 0,\ $ and\\
            2. $\vec{F}_{ext} = 0,\ $ then:\\
            $\displaystyle \der{}{t}(\vec{p}_0 + \vec{p}_f) = 0$\\
            $\displaystyle \sum_{j = 1}^{N}\vec{p}_j = \text{constant}$\vspace{2mm}\\
            $\vec{v}_{cm,f} = \vec{v}_{cm,0}$
        \end{minipage}
        };
        \draw[-] (-2.75,-1.75)--(2.75,-1.75);        
        \node[fancytitle,right=10pt,fill=gbgreen] at (box.north west) {Conservation of Momentum};
    \end{tikzpicture}

    %%% CENTER OF MASS %%%
    \begin{tikzpicture}
        \node[mybox] (box){
        \begin{minipage}{0.25\textwidth}
            \bf{System of particles:}\\
            $\displaystyle \vec{r}_{cm} = \frac{1}{M}\sum_{j = 1}^{N}m_j\vec{r}_j$\\
            $\displaystyle v_{cm} = \der{\vec{r}_{cm}}{t} = \frac{1}{M}\sum_{j = 1}^{N}m_j\der{\vec{r}_j}{t}$\\
            $M\vec{v}_{cm} = \sum_{j = 1}^{N}m_j\vec{v}_j$\\
            \bf{Continuous object:}\\
            $\displaystyle \vec{r}_{cm} = \frac{1}{M}\int\vec{r}dm$
        \end{minipage}
        };
        \draw[-] (-2.75,-1)--(2.75,-1);
        \node[fancytitle,right=10pt,fill=gbyellow] at (box.north west) {Center of Mass};
    \end{tikzpicture}
    \vspace{0.4mm}

    %%% ROTATIONAL VARIABLES %%%
    \begin{tikzpicture}
        \node[mybox] (box){
        \begin{minipage}{0.25\textwidth}
            $\displaystyle \theta = \frac{s}{r}$\\
            $\displaystyle \omega = \frac{v_t}{r} = \der{\theta}{t}$\\
            $\displaystyle \alpha = \frac{a_t}{r} = \der{\omega}{t}$
        \end{minipage}
        };
        \node[fancytitle,right=10pt,fill=gbblue] at (box.north west) {Rotational Variables};
    \end{tikzpicture}
    \vspace{0.4mm}

    %%% ROTATIONAL KINEMATICS %%%
    \begin{tikzpicture}
        \node[mybox] (box){
        \begin{minipage}{0.25\textwidth}
            $\displaystyle \theta_f = \theta_0 + \omega t$\\
            $\displaystyle \omega_f = \omega_0 + \alpha t$\\
            $\displaystyle \theta_f = \theta_0 + \omega_0t + \frac{1}{2}\alpha t^2$\\
            $\displaystyle \omega_f^2 = \omega_0^2 + 2\alpha(\Delta\theta)$
        \end{minipage}
        };
        \node[fancytitle,right=10pt,fill=gbpurple] at (box.north west) {Rotational Kinematics};
    \end{tikzpicture}
    \vspace{0.4mm}

    %%% MOMENT OF INERTIA %%%
    \begin{tikzpicture}
        \node[mybox] (box){
        \begin{minipage}{0.25\textwidth}
            $\displaystyle I = \sum_{i} m_ir_i^2$\vspace{0.1mm}\\
            $\displaystyle I = \int r^2dm$\\
            $\displaystyle I_{tot} = \sum_{i}I_i$
        \end{minipage}
        };
        \node[fancytitle,right=10pt] at (box.north west) {Moment of Inertia};
    \end{tikzpicture}
    \vspace{0.4mm}

    %%% PARALLEL AXIS THEOREM %%%
    \begin{tikzpicture}
        \node[mybox] (box){
        \begin{minipage}{0.25\textwidth}
            $I_{pa} = I_{cm} + md^2$
        \end{minipage}
        };
        \node[fancytitle,right=10pt,fill=orange] at (box.north west) {Parallel Axis Theorem};
    \end{tikzpicture}
    \vspace{0.4mm}

    %%% TORQUE %%%
    \begin{tikzpicture}
        \node[mybox] (box){
        \begin{minipage}{0.25\textwidth}
            $\tau = \vec{r} \times \vec{F} = rF\sin(\theta) = I\alpha$\\
            $\displaystyle \tau_{net} = \sum_{i}\tau_i = I\alpha$\\
            $\displaystyle W = \int\sum\vec{\tau} \cdot d\vec{\theta}$\vspace{1mm}\\
            $\displaystyle W_{AB} = \int_{\theta_a}^{\theta_B}\bigg(\sum_{i}\tau_i\bigg)d\theta$
        \end{minipage}
        };
        \node[fancytitle,right=10pt,fill=gbred] at (box.north west) {Torque};
    \end{tikzpicture}
    \vspace{0.4mm}

    %%% ANGULAR MOMENTUM %%%
    \begin{tikzpicture}
        \node[mybox] (box){
        \begin{minipage}{0.25\textwidth}
            $\vec{l} = \vec{r} \times \vec{p}\ \to\ l = rp\sin(\theta)$\\
            $L = I\omega$
            \vspace{2mm}\\
            $\displaystyle \der{\vec{L}}{t} = \sum_{i}\der{\vec{l}_i}{t} = \sum_{i}\tau_i$\vspace{2mm}\\
            \bf{Conservation of }\boldmath{$\vec{L}$}\unboldmath\\
            If: $\displaystyle \sum\vec{\tau} = 0$\\
            Then: $\displaystyle \der{\vec{L}}{t} = 0\ \to\ I_f\omega_f = I_0\omega_0$, 
        \end{minipage}
        };
        \draw[-] (-2.75,-0.15)--(2.75,-0.15);
        \node[fancytitle,right=10pt,fill=gbgreen] at (box.north west) {Angular Momentum};
    \end{tikzpicture}
\end{multicols*}
\end{center}
\end{document}

%\begin{tikzpicture}
%    \node[mybox] (box){
%    \begin{minipage}{0.25\textwidth}
%        
%    \end{minipage}
%    };
%    \node[fancytitle,right=10pt] at (box.north west) {};
%\end{tikzpicture}