\documentclass[a4paper]{article}
\usepackage[english]{babel}
\usepackage[utf8]{inputenc}
\usepackage{textcomp}
\usepackage{amsmath}
\usepackage{gensymb}
\usepackage{physics}
\usepackage{graphicx}
\usepackage[colorinlistoftodos]{todonotes}
\usepackage{xcolor}
\usepackage{array}
\usepackage{tabularx}
\usepackage{tikz}
\usepackage{pgfplots}
\usepackage{framed}
\usepackage{xfrac}
\usepackage[most]{tcolorbox}
\usepackage{fix-cm}
\usepackage{cancel}
\usepackage[margin=0.5in]{geometry}
\usetikzlibrary{quotes,angles}
\usetikzlibrary{decorations.pathreplacing}
\usetikzlibrary{calc}
\usepgfplotslibrary{fillbetween}

\let\phi\varphi
\let\bf\textbf
\colorlet{shadecolor}{orange!15}
\pgfplotsset{compat=1.18}
\newcommand\der[2]{\frac{d #1}{d #2}}
\newcommand\Deltat{\Delta t}
\newcommand\rads{\text{ rad\;s}^{-1}}
\newcommand\radss{\text{ rad\;s}^{-2}}
\newcommand\rad{\text{ rad}}
\newcommand\s{\text{ s}}
\newcommand\m{\text{ m}}
\newcommand\J{\text{ J}}
\newcommand\Nm{\text{ Nm}}
\newcommand\ms{\text{ ms}^{-1}}
\newcommand\mss{\text{ ms}^{-2}}
\newcommand\kg{\text{ kg}}
\newcommand\kgms{\text{ kg\;ms}^{-1}}
\newcommand\kgmm{\text{ kg\;m}^{2}}
\newcommand{\AxisRotator}[1][rotate=0]{%
    \tikz [x=0.25cm,y=0.60cm,line width=.2ex,-stealth,#1] \draw (0,0) arc (-150:150:1 and 1);%
}
\def\centerarc[#1](#2)(#3:#4:#5){\draw[#1] ($(#2)+({#5*cos(#3)},{#5*sin(#3)})$) arc (#3:#4:#5)}
% Syntax: \centerarc[draw options] (center) (initial angle:final angle:radius);

\begin{document}
\section{7/23}
\bf{Review: Static Equilibrium}\\
For a particle
\begin{itemize}
    \item $F_{net} = 0$
\end{itemize}
For an extended object
\begin{itemize}
    \item $F_{net} = 0$
    \item $\tau_{net} = 0$
\end{itemize}
\begin{align*}
    \tau = r\times F = I\alpha
\end{align*}
For linear momentum $\vec{P} = m\vec{v}$, for angular momentum, $\vec{L} = I\omega$
\begin{align*}
    \vec{\tau} = r\times F\\
    \vec{L} = \vec{r} \times \vec{P}\\
    L = \vec{r}\times m\vec{v} = mvrsin(\theta)\\
    L_{max} = mvr\ (v = r\omega)\\
    = (mr\times r)\omega = I\vec{omega}
\end{align*}
\begin{shaded}
    \underline{\bf{Practice:} Rotating Disk}
    What is the angular momentum about the axle of a 2.0 kg, 4.0 cm diameter disk rotating at 600 rpm\\
    Known: $m, d, \omega$, Want: L
    \begin{align*}
        L = I\omega\\
        I = \frac{1}{2}mR^2\\
        I = \frac{1}{2}m(\sfrac{d}{2})^2\\
        I = \frac{1}{2}m\frac{d^2}{4} = \frac{1}{8}md^2
        L = \frac{1}{8}md^2\omega\\
        L = 0.025\kgmm\text{/s}
    \end{align*}
\end{shaded}
\begin{align*}
    \vec{\tau}_{net} = I\alpha = \sum\tau_i = \sum\vec{r}_i\times \vec{F}_i
\end{align*}
In the absence of external torques: $\displaystyle I\alpha = \frac{d(I\vec{\omega})}{dt} = 0$, $I\vec{\omega} = L =$ constant

\begin{shaded}
    \underline{\bf{Example:} Krunchy on a Turntable}
    \vspace{2mm}\\
    Krunchy of mass $m$ rides on a disk of mass $6m$ and radius $R$ as shown. The disk rotates bround its central axis at angular speed 1.5$\rads$
\end{shaded}

\end{document}