\documentclass[a4paper]{article}
\usepackage[english]{babel}
\usepackage[utf8]{inputenc}
\usepackage{textcomp}
\usepackage{amsmath}
\usepackage{gensymb}
\usepackage{physics}
\usepackage{graphicx}
\usepackage[colorinlistoftodos]{todonotes}
\usepackage{xcolor}
\usepackage{array}
\usepackage{tabularx}
\usepackage{tikz}
\usepackage{pgfplots}
\usepackage{framed}
\usepackage{xfrac}
\usepackage[most]{tcolorbox}
\usepackage{fix-cm}
\usepackage{cancel}
\usepackage[margin=0.5in]{geometry}
\usetikzlibrary{quotes,angles}
\usetikzlibrary{decorations.pathreplacing}
\usetikzlibrary{calc}
\usepgfplotslibrary{fillbetween}

\let\phi\varphi
\let\bf\textbf
\colorlet{shadecolor}{orange!15}
\pgfplotsset{compat=1.18}
\newcommand\der[2]{\frac{d #1}{d #2}}
\newcommand\Deltat{\Delta t}
\newcommand\rads{\text{ rad\;s}^{-1}}
\newcommand\radss{\text{ rad\;s}^{-2}}
\newcommand\rad{\text{ rad}}
\newcommand\s{\text{ s}}
\newcommand\m{\text{ m}}
\newcommand\J{\text{ J}}
\newcommand\Nm{\text{ Nm}}
\newcommand\ms{\text{ ms}^{-1}}
\newcommand\mss{\text{ ms}^{-2}}
\newcommand\kg{\text{ kg}}
\newcommand\kgms{\text{ kg\;ms}^{-1}}
\newcommand\kgmm{\text{ kg\;m}^{2}}
\newcommand{\AxisRotator}[1][rotate=0]{%
    \tikz [x=0.25cm,y=0.60cm,line width=.2ex,-stealth,#1] \draw (0,0) arc (-150:150:1 and 1);%
}
\def\centerarc[#1](#2)(#3:#4:#5){\draw[#1] ($(#2)+({#5*cos(#3)},{#5*sin(#3)})$) arc (#3:#4:#5)}
% Syntax: \centerarc[draw options] (center) (initial angle:final angle:radius);

\begin{document}
\section{7/23}
\bf{Review: Static Equilibrium}\\
For a particle
\begin{itemize}
    \item $F_{net} = 0$
\end{itemize}
For an extended object
\begin{itemize}
    \item $F_{net} = 0$
    \item $\tau_{net} = 0$
\end{itemize}
\begin{align*}
    \tau = r\times F = I\alpha
\end{align*}
For linear momentum $\vec{P} = m\vec{v}$, for angular momentum, $\vec{L} = I\omega$
\begin{align*}
    \vec{\tau} = r\times F\\
    \vec{L} = \vec{r} \times \vec{P}\\
    L = \vec{r}\times m\vec{v} = mvrsin(\theta)\\
    L_{max} = mvr\ (v = r\omega)\\
    = (mr\times r)\omega = I\vec{omega}
\end{align*}
\begin{shaded}
    \underline{\bf{Practice:} Rotating Disk}
    What is the angular momentum about the axle of a 2.0 kg, 4.0 cm diameter disk rotating at 600 rpm\\
    Known: $m, d, \omega$, Want: L
    \begin{align*}
        L = I\omega\\
        I = \frac{1}{2}mR^2\\
        I = \frac{1}{2}m(\sfrac{d}{2})^2\\
        I = \frac{1}{2}m\frac{d^2}{4} = \frac{1}{8}md^2
        L = \frac{1}{8}md^2\omega\\
        L = 0.025\kgmm\text{/s}
    \end{align*}
\end{shaded}
\begin{align*}
    \vec{\tau}_{net} = I\alpha = \sum\tau_i = \sum\vec{r}_i\times \vec{F}_i
\end{align*}
In the absence of external torques: $\displaystyle I\alpha = \frac{d(I\vec{\omega})}{dt} = 0$, $I\vec{\omega} = L =$ constant

\begin{shaded}
    \underline{\bf{Example:} Krunchy on a Turntable}
    \vspace{2mm}\\
    Krunchy of mass $m$ rides on a disk of mass $6m$ and radius $R$ as shown. The disk rotates bround its central axis at angular speed 1.5$\rads$
\end{shaded}
\newpage

\section{7/24}
Final will be in class, online, no work submitted, no partial credit (Do lots of MOI calculations to prepare for final)
\begin{shaded}
    \underline{\bf{Example:} Putty on a Turntable}
    \vspace{2mm}\\
    A small blob of putty of mass $m$ falls from the ceiling and lands on the outer rim of a turntable of radius $R$ and moment of inertia $I_0$ that is rotating freely with angular speed $\omega_0$ about a vertical axis passing through the center of the turntable and perpendicular to the surface of the turntable
    \begin{enumerate}
        \item What is the post-collision angular speed of the turntable-putty system?
        \item After several turns, the blob flies off the edge of the turntable. What is the angular speed of the turntable after the blob's departure?
        \begin{enumerate}       
            \item $I_{d,cm} = I_0,\ \omega_d = \omega_0$
            \item $I_{mass} = mR^2,\ \omega' = ?,\ \tau_{ext} = 0$\\
            $L_i = L_f\ \to\ \omega_0 = (I_0 + mR^2)$
            \item $v = R\omega'$\\
            $L = L_{I_0} + L_m$\\
            $R\times P \to L = Rmv\ L_m = RmR\omega'$\\
            $L_3 = I_0\omega'' + (mR^2)\omega'$\\
            $L_2 = L_3$\\
            $(I_0 + mR^2)\omega' = I_0\omega'' + (mR^2)\omega'$\\
            $I_0\omega' = I_0\omega''$\\
            $\omega' = \omega'' \neq \omega_0$
        \end{enumerate}
    \end{enumerate}
\end{shaded}
\begin{shaded}
    \underline{\bf{Example:} Spinning Disks}
    \vspace{2mm}\\
    Two uniform disks with masses $M_1 = 2\kg$, $M_2 = 5\kg$, and radii $R_1 = 0.10\m,\ R_2 = 0.15\m$ are spinning freely about their center axis at frequencies $f_1 = 1200$ rpm, and $f_2 = 1500$ rpm. The cylinders are brought together and come to the same angular velocity via frictional contact. The moment of inertia of a uniform cylinder is given by $I = \frac{1}{2}MR^2$
    \begin{enumerate}
        \item[A.] Find the angular speed of each cylinder before they are joined\\
        $\omega = 2\pi f$\\
        $\omega_1 = 2\pi(1200$ rev/min$)(1$ min/60 s$) = 126\rads$\\
        $\omega_2 = 2\pi(1500$ rev/min$)(1$ min/60 s$) = 157\rads$
        \item[B.] Find the total kinetic energy of the two cylinders before they are joined\\
        $M_1 = 2\kg,\ R_1 = 0.1\m,\ \omega_1 = 126\rads$ $M_2 = 5\kg,\ R_2 = 0.15\m,\ \omega_2 = 157\rads$\\
        $L_i = L_f$\\
        $L_{i,1} + L_{i,2} = L_f$\\
        $I_1\omega_1 + I_2\omega_2 = (I_1 + I_2)\omega_f$\\
        $\frac{1}{2}M_1R_1^2\omega_1 + \frac{1}{2}M_2R_2^2\omega_2 = (\frac{1}{2}M_1R_1^2 + \frac{1}{2}M_2R_2^2)\omega_f$\\
        $\omega_f = \frac{\frac{1}{2}M_1R_1^2\omega_1 + \frac{1}{2}M_2R_2^2\omega_2}{\frac{1}{2}M_1R_1^2 + \frac{1}{2}M_2R_2^2}$\\
        $\frac{1}{2}M_1R_1^2\omega_1 = 1.26\ \frac{1}{2}M_2R_2^2\omega_2 = 8.83\ \frac{1}{2}M_1R_1^2 = 0.1$
        \item[C.] Find the angular speed of each cylinder after they couple
    \end{enumerate}
\end{shaded}

\end{document}